Three datasets were used in the test of the effects of country and government mandates. These three datasets included
data on the dates of different quarantine measures per country \cite{govtresponse},  the time series data for the case numbers \cite{owid} and the time series data on the number of tests \cite{find}. These datasets are quite inconsistent due to differences in reaction to the pandemic and the inconsistency of reporting. This manifests as irregular time-series, in terms of the dates on which they are defined, as well as missing values. 
Because of this, a number of actions which may or may not be heavy handed needed to be employed.  
These actions accounted for the following: differences in time series ranges, missing values, quarantine measures which occur before the first recorded case in a country, dissimilar quarantine measures taken and differences in the countries whose data was recorded. 
To make the data uniform, the time series were normalized to be from December 31st 2019 to April 20th, 2020. The end date was simply because one of the data sets has not been updated in the past week. The original missing values and those created by this normalization were handled by using linear interpolation on the interior of the time series and linear extrapolation for the beginning of the time series. This is only applied at the very beginning of each countries time series (if values are missing) typically when the cases number is very small or even equal to 1. Therefore, I believe this action is justified by using linear expansion of exponential growth. After the time series were made uniform, the specific government responses to investigate
were chosen by how widespread their adoption was. The countries were determined by taking the intersection of all countries present in the three data sets.