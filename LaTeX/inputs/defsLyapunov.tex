% siminos/inputs/defsLyapunov.tex
% $Author: xiong $ $Date: 2016-04-14 08:54:09 -0400 (Thu, 14 Apr 2016) $

%\newcommand{\YTlink}[1]{\marginpar{   %% links to YouTube videos
%    \HREF{http://#1}{\includegraphics[width=16pt]{start-32}}}
%                       }
\newcommand{\YTlink}[1]{    %% in text, links to YouTube videos
    \raisebox{-0.4ex}[0pt][0pt]{\!\!\!
    \HREF{http://#1}{\includegraphics[height=1em]{start-32}}
    \!\!\!            }  }

\newcommand{\XD}{Xiong Ding}

\ifpaper % prepare for B&W paper printing:
       \renewcommand{\arXiv}[1]{ {arXiv:#1}}
\else % prepare hyperlinked pdf
       \renewcommand{\arXiv}[1]{
              {\href{http://arXiv.org/abs/#1}{arXiv:#1}}}
\fi

    \ifboyscout
\newcommand{\toCB}{\marginpar{\footnotesize 2CB}}  % to compare with ChaosBook
\newcommand{\inCB}{\marginpar{\footnotesize now in CB}} % entered in ChaosBook
    \else
\newcommand{\toCB}{}
\newcommand{\inCB}{}
    \fi %end of internal draft switch

%%%%%%%%%%%% MACROS, Xiong Ding specific %%%%%%%%%%
\newcommand{\xDft}{ Discrete Fourier Transform }
\newcommand{\cLv} {covariant vector}  % ChaosBook.org
%\newcommand{\cLv} {covariant Lyapunov vector} % Ginelli et al
\newcommand{\CLv} {Covariant vector}  % ChaosBook.org
%\newcommand{\CLv} {Covariant Lyapunov vector} % Ginelli et al
\newcommand{\cLvs} {covariant vectors}  % ChaosBook.org
%\newcommand{\cLvs} {covariant Lyapunov vectors} % Ginelli et al
\newcommand{\CLvs} {Covariant vectors}  % ChaosBook.org
%\newcommand{\CLvs} {Covariant Lyapunov vectors} % Ginelli et al
\newcommand{\transient}{transient} % ChaosBook.org
% \newcommand{\transient} {spurious} % Kaz uses {spurious}
\newcommand{\entangled}{entangled}  % ChaosBook.org
% \newcommand{\entangled} {physical} % Kaz uses {physical}
\newcommand{\Entangled}{Entangled}  % ChaosBook.org
\newcommand{\psd}{Periodic Schur Decomposition}
\newcommand{\prsf}{Periodic Real Schur Form}
\newcommand{\LMa}{Levenberg-Marquardt algorithm}
\newtheorem{per_schur}{Theorem}
\newcommand{\pse}{Periodic Sylvester Equation}
\newcommand{\psm}{Periodic Sylvester Matrix}
\newcommand{\ped}{Periodic Eigendecomposition}
\newcommand{\Ped}{Periodic Eigendecomposition}
\newcommand{\ps}[2]{\mathbf{#1}^{(#2)}}
\newcommand{\pqr}{periodic QR algorithm}
\newcommand{\Pqr}{Periodic QR algorithm}
\newcommand{\Fv}{Floquet vector}
\newcommand{\Fe}{Floquet exponent}
\newcommand{\Fm}{Floquet multiplier}
%\newcommand{\ppo}{pre-periodic orbit}
%\newcommand{\Ppo}{Pre-periodic orbit}

\newcommand{\cGL}{complex Ginzburg-Landau}
\newcommand{\CGL}{Complex Ginzburg-Landau}
\newcommand{\cGLe}{complex Ginzburg-Landau equation}
\newcommand{\CGLe}{Complex Ginzburg-Landau equation}
\newcommand{\cqcGL}{cubic quintic complex Ginzburg-Landau}
\newcommand{\CqcGL}{Cubic quintic complex Ginzburg-Landau}
\newcommand{\cqcGLe}{cubic quintic complex Ginzburg-Landau equation}
\newcommand{\CqcGLe}{Cubic quintic complex Ginzburg-Landau equation}

\newcommand{\inm}{inertial manifold}
\newcommand{\Inm}{Inertial manifold}
\newcommand{\InM}{Inertial Manifold}
%%%%%%%%%%%% MACROS, Lippolis thesis specific %%%%%%%%%%
\newcommand{\optPart}{optimal partition}
\newcommand{\OptPart}{Optimal partition}

%%%%%%%%%%%% MACROS, Siminos specific %%%%%%%%%%
\newcommand{\edit}[1]{{\color{blue} #1}} % for referees
%\newcommand{\edit}[1]{#1}               % for the journal
\newcommand{\vf}{v}	%%% keep notation for vector field flexible.
                    % For the time being follow Das Buch.
\newcommand{\conj}[1]{\ensuremath{\bar{#1}}}
\newcommand{\Manif}{\ensuremath{\mathcal{M}}}
\newcommand{\Order}[1]{\mathrm{O}(#1)}
\newcommand{\steady}{\textdollar~}

%%%%%%%% Symmetries, Siminos specific %%%%%%%%%%
%\renewcommand{\shift}{\ensuremath{\ell}}
%\newcommand{\Shift}{\ensuremath{\tau}}
\newcommand{\Idg}{\ensuremath{\mathbf{1}}}
\newcommand{\Cn}[1]{\ensuremath{\mathrm{C}_{#1}}}
\newcommand{\En}[1]{\ensuremath{\mathrm{E}(#1)}}
\newcommand{\Rotn}[1]{\ensuremath{R_{#1}}}
\newcommand{\Drot}{\ensuremath{\zeta}}
\newcommand{\globstab}[1]{\ensuremath{\Sigma_{#1}}} % Change to be the same as stab. Was \Sigma^\ast_{#1}
\newcommand{\Str}[1]{\ensuremath{\mathcal{S}_{#1}}} % Stratum
\newcommand{\Nlz}[1]{\ensuremath{N(#1)}}
\newcommand{\doubleperiod}[1]{{\ensuremath{\mathcal{T}_{#1}}}}

%%%%%%%%%%%% Theorems, Siminos specific %%%%%%%%%%
\newtheorem{definition}{Definition}[chapter]

\newtheorem{theorem}[definition]{Theorem}
\newtheorem{lemma}[definition]{Lemma}
\newtheorem{proposition}[definition]{Proposition}
% \newtheorem{example}[definition]{Example}

\newcommand{\refLem}[1]{Lemma~\ref{#1}}
\newcommand{\refThe}[1]{Theorem~\ref{#1}}
\newcommand{\refPro}[1]{Proposition~\ref{#1}}
% \newcommand{\refExa}[1]{Example~\ref{#1}}
\newcommand{\refdef}[1]{definition~\ref{#1}}

%%%%%%%%%%%%%% Solution labels %%%%%%%%%%%%%%%%%%%%
\newcommand{\EQB}[1]{\ensuremath{\mathrm{E}_{#1}}}
\newcommand{\REQB}[1]{\ensuremath{\mathrm{Q}_{#1}}} % For ODE's, use REQV from chaosbook for PDE's

% Redefine using mathrm, it is a label not a math symbol
\renewcommand{\EQV}[1]{\ensuremath{\mathrm{E}_{#1}}}
% E_0: u = 0 - trivial equilibrium
% E_1,E_2,E_3, for 1,2,3-wave equilibria
\renewcommand{\REQV}[2]{\ensuremath{\mathrm{TW}_{#1#2}}} % #1 is + or -
% TW_1^{+,-} for 1-wave traveling waves (positive and negative velocity).
\renewcommand{\PO}[1]{\ensuremath{\mathrm{PO}_{#1}}}
% PO_{period to 2-4 significant digits} - periodic orbits
\renewcommand{\RPO}[1]{\ensuremath{\mathrm{RPO}_{#1}}}
% RPO_{period to 2-4 significant digits} - relative PO.  We use ^{+,-}
% to distinguish between members of a reflection-symmetric pair.
% Gibson likes:
\renewcommand{\tEQ}{\ensuremath{\mathrm{EQ}}}

\newcommand{\tLB}{\ensuremath{{\text{EQ}_1}}}
\newcommand{\tUB}{\ensuremath{{\text{EQ}_2}}}

%%%%%%%%%%%%%% Operators %%%%%%%%%%%%%%%%%%%%%%%
\newcommand{\PperpOp}{\mathbf{P}^{\perp}}
%\newcommand{\Pperp}{P^{\perp}}
%

%%%%%%%%%%% Equations, Siminos specific %%%%%%%%%%
\newcommand{\cont}{\,, \\ }

%%%%%%%%%%%% REMOVE THIS EVENTUALLY %%%%%%%%%%%
%\newcommand\PoincSec{Poincar\'e section}

	% without large brackets:
%\newcommand{\braket}[2]
%		   {\langle{#1}\vphantom{#2}|\vphantom{#1}{#2}\rangle}
%\newcommand{\bra}[1]{\langle{#1}\vphantom{ }|}
%\newcommand{\ket}[1]{|\vphantom{}{#1}\rangle}
	% with large brackets:
%\newcommand{\bra}[1]{\left\langle{#1}\vphantom{ }\right|}
%\newcommand{\ket}[1]{\left|\vphantom{}{#1}\right\rangle}
%\newcommand{\braket}[2]{\left\langle{#1}
%                        \vphantom{#2}\right|\left.\vphantom{#1}
%                        {#2}\right\rangle}

%\renewcommand{\Lg}{\ensuremath{T}}   % FrCv11.tex Lie algebra generator

%\newcommand{\dual}[1]{{#1}^T}		% SO(n) case
%\newcommand{\Sset}{Inflection hyperplane}
%\newcommand{\sset}{inflection hyperplane} 	% {singularity hyperplane}
%											% {singular set}
%\newcommand{\sspSing}{\ensuremath{\ssp^*}} 	% inflection point
%\newcommand{\sspRSing}{\ensuremath{\sspRed^*}} 	% inflection point, reduced space
%\newcommand{\template}{template} % {slice-fixing point} % {reference state}
