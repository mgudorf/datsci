\subsection{Mike Badescu(1) Saturday, September 7th 2019}

Notes regarding the first meeting with Mike Badescu,
Romanian PhD who now does data science.

Mainly this a record of the advice he gives for starting a career
in the data science field. In terms of extracurriculars'
he mainly thinks it is better to finish the code (datacamp)
as fast as possible so that I can focus on more projects. 
This includes jumping ahead to chapters seven, eight, 
ten and twelve to get a better
feel for the machine learning aspect of the course so that
the first capstone can be started asap.

Advice for the capstone project, choose data that
is interesting and that you like. Sometimes its
best to give yourself an ultimatum 
as if the boss is given you a adata set and is telling you to work
on it in a mandatory fashion. 

(Personal advice, start looking through tensorflow and 
scikit-learn) and make very basic scripts).

Back to capstone advice, code in some kind
of data cleaning
procedures because thats most of the difficulty
in the real world. Unit tests, edge cases, corner
cases. 



Other advice (kind of regarding interviews) is
to leverage everything possible including making
analogies of teaching to team experience. 


Videos of students going over projects, and
sample projects. Brainstorm random stuff,
just ridiculous hypotheses to get some kind
of idea. 

Need to finish unit two and three by wednesday,
and find a data set for the capstone project.

Make LinkedIn, Github, stackoverflow all
have the same account name and picture so
that google finds you easier.

Coming from a non-CS background is okay
because both people developing the theory 
as well as people doing more applied sciences
is necessary, and actually the applied part might
be more useful in industry.

Mike has an atypical industry career so
perhaps talk to other mentors that have
more traditional industry experience.

Sell Physics work as a story, really
try to relate content to the people
interviewing as well as the topics
also try to sell Physics ideas to people that hate Physics. 

For code, put on a docker container
with a predefined (i.e. not local or custom 
environment) environment. Look
at source code from other packages to get
a sense how things are setup.


For data science in general some good tips
on how to start any problem is to do
quick, likely crude visualizations,
graphs,correlations, skewness, transformations
that might help data viewing or trend analysis.
Standard steps is to clean data, with unit tests
etc. Give clients ideas that the might not
have thought of to produce more work for yourself.
The worst thing to happen is that you CAN perform
some type of analysis but the results arent there.

Really stressed taking advantage of Springboard
resources, other mentors, information.

