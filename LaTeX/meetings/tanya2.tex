\subsection{Tanya Kuwadekar(1) Thursday, January 17th 2019}

\subsubsection{What to do at the meetup}
Just talk to people,

Ask about other meetups
Try to extract information of other meetups.
Just try to get a feel for the data science community itself.

What sort of topics?
where do they work?
what projects are they working on?
Try to ask the speaker, make definite connections.

Be up front with my goals. Not
directly bugging for jobs, just find relevant.

With data scientists might not be able to contribute
to a knowledge perspective; just leverage the advice,
pseudo-mentor. How best to position myself in the data
science.

Resources, tools, technologies, what I should focus on.
Give a chance to let people help you. 

Fresh perspective on the industry, might learn about
people are learning nowadays. They'll get a chance to help
you. The business world is not as scary. 

Goals in mind: Reaching out with a set number of people.
Doesn't have to be data scientist, just have casual conversations
and get there names. Connect through LinkedIn.
The initial conversation, look into profile, make a connection
and then ask for advice. 

Can choose how to deliver the message but cannot choose how the
message is received. 
How long after does the connection go cold?

Maybe start with a meet-up unrelated to data science. 
Go to a meet-up every month. 

Try to connect with three people, at least the speaker.


\subsubsection{How to cold e-mail people}
twenty to thirty percent, really go for the advice angle.
The 

This seems too reliant on other people/brown nosy
for my taste.


The reasons for networking and cold e-mailing is to get
advice; how people are using Data Science in your company.
At this point we're only 36 percent complete.

\subsubsection{LinkedIn}
Social media so more expectations. Linked in might have more
content, its the same exact process as the resume.




