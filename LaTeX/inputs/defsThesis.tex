% siminos/inputs/defsThesis.tex
% $Author: xiong $ $Date: 2017-02-02 00:40:16 -0500 (Thu, 02 Feb 2017) $

    \ifboyscout
\newcommand{\toCB}{\marginpar{\footnotesize 2CB}}  % to compare with ChaosBook
\newcommand{\inCB}{\marginpar{\footnotesize now in CB}} % entered in ChaosBook
\newcommand{\toMapit}{\marginpar{\footnotesize 2mapit}}  % to compare with WFSBC15
\newcommand{\inMapit}{\marginpar{\footnotesize now in mapit}} % entered in WFSBC15
    \else
\newcommand{\toCB}{}
\newcommand{\inCB}{}
\newcommand{\toMapit}{}
\newcommand{\inMapit}{}
    \fi %end of internal draft switch

%%%%%%%%%%%% Budanur thesis definitions %%%%%%%%%%%%%%%
\renewcommand{\ssp}{a} %State space vector
\newcommand{\Fu}{\tilde{u}} %Fourier transform of a scalar field
\newcommand{\FuRed}{\hat{u}} %Symmetry reduced Fourier mode
%\newcommand{\twomode}{two-mode} %Fourier transform of a scalar field
%\newcommand{\twoMode}{Two-mode} %Fourier transform of a scalar field
\renewcommand{\obser}{\omega}      % an observable from state space to R^n
\renewcommand{\Obser}{\Omega}      % time integral of an observable
\newcommand{\matrixRep}{\ensuremath{{D}}}  %  matrix rep of a group element
\newcommand{\conf}{\ensuremath{x}} %Configuration space coordinate
\newcommand{\OpSlice}{\ensuremath{\mathbf{S}}}
%\newcommand{\fFslice}{first Fourier mode slice}
%\newcommand{\FFslice}{First Fourier mode slice}
\renewcommand{\zeit}{\ensuremath{\tau}}
%\newcommand{\slicePlane}{slice hyperplane}
%\newcommand{\SlicePlane}{Slice hyperplane}
\renewcommand{\Lg}{\ensuremath{T}}
%\newcommand{\MvarRed}{\hat{\Mvar}}
\newcommand{\NS}{Navier-Stokes}
\newcommand{\NSe}{Navier-Stokes equations}
\newcommand{\LieElz}[1]{\ensuremath{\LieEl_z (#1)}}  % z-shifts notation
\newcommand{\inprod}[2]{\langle #1 ,  #2 \rangle}
%Two-mode macros:
% Coordinate systems
\newcommand{\cartpt}[1]{\left( #1 \right)} % Points in full Cartesian 4D space
\newcommand{\polpt}[1]{$\left\{ #1 \right\}$} % Points in polar coordinates
\newcommand{\invpt}[1]{\left[ #1 \right]} % Points in invariant polynomial basis
\newcommand{\sspC}{\ensuremath{z}} %Complex valued state space variable
\newcommand{\sspRedC}{\ensuremath{\hat{\sspC}}}
\newcommand{\oneMinJred}[1]
           {\left|\det\!\left(\matId-\monodromyRed_p^{#1}\right)\right|}
\newcommand{\invpol}{\ensuremath{p}} % Invariant polynomials
\newcommand{\monodromyRed}{\ensuremath{\hat{\monodromy}}}   % monodromy matrix, full Poincare cut

%%%%%%%%%%%% Xiong Ding thesis definitions %%%%%%%%%%%%%%%
\newcommand{\XD}{Xiong Ding}
\newcommand{\xDft}{ Discrete Fourier Transform }
\newcommand{\cLv} {covariant vector}  % ChaosBook.org
%\newcommand{\cLv} {covariant Lyapunov vector} % Ginelli et al
\newcommand{\CLv} {Covariant vector}  % ChaosBook.org
%\newcommand{\CLv} {Covariant Lyapunov vector} % Ginelli et al
\newcommand{\cLvs} {covariant vectors}  % ChaosBook.org
%\newcommand{\cLvs} {covariant Lyapunov vectors} % Ginelli et al
\newcommand{\CLvs} {Covariant vectors}  % ChaosBook.org
%\newcommand{\CLvs} {Covariant Lyapunov vectors} % Ginelli et alLagrangePerturbed
\newcommand{\entangled} {physical}
\newcommand{\transient} {spurious}
\newcommand{\psd}{periodic Schur decomposition}
\newcommand{\Psd}{Periodic Schur decomposition}
\newcommand{\prsf}{Periodic Real Schur Form}
\newcommand{\LMa}{Levenberg-Marquardt algorithm}
\newtheorem{per_schur}{Theorem}
\newcommand{\pse}{periodic Sylvester equation}
\newcommand{\Pse}{Periodic Sylvester equation}
\newcommand{\psm}{Periodic Sylvester Matrix}
\newcommand{\pqr}{periodic QR algorithm}
\newcommand{\Pqr}{Periodic QR algorithm}
\newcommand{\ped}{periodic eigendecomposition}
\newcommand{\Ped}{Periodic eigendecomposition}
\newcommand{\ps}[2]{\mathbf{#1}^{(#2)}}


\newcommand{\cGL}{complex Ginzburg-Landau}
\newcommand{\CGL}{Complex Ginzburg-Landau}
\newcommand{\cGLe}{complex Ginzburg-Landau equation}
\newcommand{\CGLe}{Complex Ginzburg-Landau equation}
\newcommand{\cqcGL}{cubic quintic complex Ginzburg-Landau}
\newcommand{\CqcGL}{Cubic quintic complex Ginzburg-Landau}
\newcommand{\cqcGLe}{cubic quintic complex Ginzburg-Landau equation}
\newcommand{\CqcGLe}{Cubic quintic complex Ginzburg-Landau equation}

\newcommand{\Rve}[1]{v_{#1}} % eigenvector of upper-triangular matrix R.
\newcommand{\Fv}{Floquet vector}
\newcommand{\ve}{\bm{e}}
\newcommand{\op}{\mathrm{\mathbf{x}}}
\newcommand{\opa}{\mathrm{\mathbf{a}}}
\newcommand{\Jve}[2][0]{\ensuremath{{\bf e}_{#2}^{(#1)}}} % right jacobiam eigenvector
\newcommand{\pMat}{h}
\newcommand{\pMatM}{\pMat^{(-)}}

%%%%%%%%%%%% MACROS, Siminos thesis specific %%%%%%%%%%

%%%%%%%%%%%%%%% REFERENCING EQUATIONS ETC. %%%%%%%%%%%%%%%%%%%%%%%%%%%%%%%
    % \renewcommand{\refref} [1] {Ref.~\cite{#1}}
    % \renewcommand{\refrefs}[1] {Refs.~\cite{#1}}
\renewcommand{\refeq}  [1] {(\ref{#1})}
\renewcommand{\refeqs} [2]{(\ref{#1}--\ref{#2})}
\renewcommand{\reffig} [1] {Figure~\ref{#1}}
\renewcommand{\reffigs} [2] {Figures~\ref{#1} and~\ref{#2}}
\renewcommand{\reftab} [1] {Table~\ref{#1}}
\renewcommand{\reftabs}[2] {Tables~\ref{#1} and~\ref{#2}}
\renewcommand{\refsect}[1] {Sect.~\ref{#1}}
\renewcommand{\refsects}[2] {Sects.~\ref{#1} and \ref{#2}}
\renewcommand{\refchap}[1] {Chapter~\ref{#1}}
\renewcommand{\refchaps}[2] {Chapters~\ref{#1} and \ref{#2}}
\renewcommand{\refchaptochap}[2] {Chapters~\ref{#1} to \ref{#2}}
\renewcommand{\refappe}[1] {Appendix~\ref{#1}}
\renewcommand{\refappes}[2] {Appendices~\ref{#1} and \ref{#2}}
\renewcommand{\refrem} [1] {Remark~\ref{#1}}
\renewcommand{\refexam}[1] {Example~\ref{#1}}
\renewcommand{\refexer}[1] {Exercise~\ref{#1}}
\renewcommand{\refsolu}[1] {Solution~\ref{#1}}

\renewcommand{\refsecttosect}[2] {Sects.~\ref{#1} to \ref{#2}}


\newcommand{\vf}{v} %%% keep notation for vector field flexible. For the time being follow Das Buch.
%\newcommand{\Lint}[1]{\frac{1}{L}\!\oint d#1\,}
\newcommand{\ode}{ODE}
% \newcommand{\Rls}[1]{\ensuremath{\mathbb{R}^{#1}}}
\newcommand{\Clx}[1]{\ensuremath{\mathbb{C}^{#1}}}
\newcommand{\conj}[1]{\ensuremath{\bar{#1}}}
\newcommand{\trace}{\mbox{\rm trace}\,}
\newcommand{\Manif}{\ensuremath{\mathcal{M}}}
\newcommand{\Order}[1]{\mathrm{O}(#1)}
%%%%%%%%%%%% Fibre bundles

\newcommand{\tSp}{E}
\newcommand{\bSp}{X}
\newcommand{\prj}{\pi}

%%%%%%%% Symmetries
% \newcommand{\On}[1]{\ensuremath{\mathrm{O}(#1)}}
\newcommand{\Rg}[1]{\Rls{#1}}
\newcommand{\Idg}{\ensuremath{\mathbf{1}}}
% \newcommand{\SOn}[1]{\ensuremath{\mathrm{SO}(#1)}}
% \newcommand{\Dn}[1]{\ensuremath{\mathrm{D}_{#1}}}
\newcommand{\Cn}[1]{\ensuremath{\mathrm{C}_{#1}}}
% \newcommand{\Zn}[1]{\ensuremath{\mathrm{Z}_{#1}}}
\newcommand{\En}[1]{\ensuremath{\mathrm{E}(#1)}}
%\newcommand{\Zn}[1]{\ensuremath{C_#1}}         % in DasBuch
%\newcommand{\Ztwo}{\ensuremath{\mathbf{Z}_2}}   % in thesis (obsolete)
% \newcommand{\Ztwo}{\ensuremath{\mathrm{D}_1}}           % in DasBuch & thesis
% \newcommand{\Refl}{\ensuremath{\kappa}}         %%%% Changed this, use R for rotations.
\renewcommand{\Shift}{\ensuremath{\tau}}
\renewcommand{\shift}{\ensuremath{\ell}}
% \newcommand{\Rot}[1]{\ensuremath{R(#1)}}
\newcommand{\Rotn}[1]{\ensuremath{R_{#1}}}
\newcommand{\Drot}{\ensuremath{\zeta}}
% \newcommand{\stab}[1]{\ensuremath{\Sigma_{#1}}}
\newcommand{\globstab}[1]{\ensuremath{\Sigma_{#1}}} % Change to be the same as stab. Was \Sigma^\ast_{#1}
\newcommand{\Str}[1]{\ensuremath{\mathcal{S}_{#1}}} % Stratum
\newcommand{\Nlz}[1]{\ensuremath{N(#1)}}
\newcommand{\doubleperiod}[1]{{\ensuremath{\mathcal{T}_{#1}}}}
\newcommand{\trDiscr}[2]{\tau_{#1}^{#2}}    % discrete cell translation 1/4, ...
%%%%%%%%%%%%%%%%%%%%%%

\newcommand{\nameit}{\ensuremath{w-}}
\newcommand{\bbUplus}{\Fix{\Dn{1}}}
\newcommand{\bbUone}{\Shift_{1/4}\Fix{\Dn{1}}}
\renewcommand{\bbU}{\mathbb{U}}
% \newcommand{\refneq}[1]{(\ref{#1})}
\newcommand{\refFigToFig}[2]{Figures~\ref{#1} to~\ref{#2}}
\newcommand{\reffigTofig}[2]{Figures~\ref{#1} to~\ref{#2}}
\newcommand{\reffigpart}[2]{Figure~\ref{#1}(#2)}
\newcommand{\refFigpart}[2]{Figure~\ref{#1}(#2)}


%%%%%%%%%%% Theorems %%%%%%%%%%%%%%%%%%%%%%%%%%%%%%%%%%
\newtheorem{definition}{Definition}[chapter]
\newtheorem{theorem}[definition]{Theorem}
\newtheorem{lemma}[definition]{Lemma}
\newtheorem{proposition}[definition]{Proposition}
\newtheorem{example}[definition]{Example}

\newcommand{\refLem}[1]{Lemma~\ref{#1}}
\newcommand{\refThe}[1]{Theorem~\ref{#1}}
\newcommand{\refDef}[1]{Definition~\ref{#1}}
\newcommand{\refPro}[1]{Proposition~\ref{#1}}
\newcommand{\refExa}[1]{Example~\ref{#1}}


%%%%%%%%% Flows

\newcommand{\AGHe}{Armbruster-Guckenheimer-Holmes flow}

%%%%%%%%%%% Equations

\newcommand{\cont}{\,, \\ }

%%%%%%%%%%%% Abbreviations, Siminos thesis specific %%%%%%

%\renewcommand{\etc}{{\em etc.}}       % etcetera in italics
%\renewcommand{\ie}{{that is}}     % use Latin or English?  Decide later.
%\renewcommand{\cf}{{\em cf.}}
%\renewcommand{\etal}{{\em et al.}}              % etcetera in italics

% PC Jul 3 2009: removed Siminos redefinions:
%\renewcommand{\statesp}{phase space}
%\renewcommand{\Statesp}{Phase space}
% PC Aug 20 2009: temporarily disabled ChaosBook definions:
\renewcommand{\reducedsp}{reduced state space}
\renewcommand{\Reducedsp}{Reduced state  space}

%\renewcommand{\slicep}{\ensuremath{\ssp^*}}   % slice-fixing point %Commented-out by Burak at 05/27/2015
%\renewcommand{\sliceTan}{\ensuremath{t^*}}    % group orbit tangent at slice-fixing %Commented-out by Burak at 05/27/2015
\renewcommand{\csection}{cross-section}
\renewcommand{\groupTan}{\ensuremath{t}}    % group orbit tangent
%\renewcommand{\Group}{\ensuremath{\Gamma}}    % Siminos Lie group
%\newcommand{\Group}{\ensuremath{G}}         % Predrag Lie or discrete group
%\newcommand{\Lg}{\mathfrak{a}}             % Siminos Lie algebra generator
%\renewcommand{\Lg}{\ensuremath{\mathbf{T}}}   % Predrag Lie algebra generator %Commented-out by Burak at 05/27/2015
%\newcommand{\LieEl}{\ensuremath{\mathbb{G}}}  % Wiczek project Lie group element
\renewcommand{\LieEl}{\ensuremath{g}}  % Predrag Lie group element

%%%%%%%%%%%%%% Solution labels %%%%%%%%%%%%%%%%%%%%
\newcommand{\EQB}[1]{\ensuremath{\mathrm{E}_{#1}}}
\newcommand{\REQB}[1]{\ensuremath{\mathrm{Q}_{#1}}} % For ODE's, use REQV from chaosbook for PDE's

% Redefine using mathrm, it is a label not a math symbol
\renewcommand{\EQV}[1]{\ensuremath{\mathrm{E}_{#1}}}
% E_0: u = 0 - trivial equilibrium
% E_1,E_2,E_3, for 1,2,3-wave equilibria
\renewcommand{\REQV}[2]{\ensuremath{\mathrm{TW}_{#1#2}}} % #1 is + or -
% TW_1^{+,-} for 1-wave traveling waves (positive and negative velocity).
\renewcommand{\PO}[1]{\ensuremath{\mathrm{PO}_{#1}}}
% PO_{period to 2-4 significant digits} - periodic orbits
\renewcommand{\RPO}[1]{\ensuremath{\mathrm{RPO}_{#1}}}
% RPO_{period to 2-4 significant digits} - relative PO.  We use ^{+,-}
% to distinguish between members of a reflection-symmetric pair.
% Gibson likes:
\renewcommand{\tEQ}{\ensuremath{\mathrm{EQ}}}


%%%%%%%%%%%%%%% From KS rpo paper %%%%%%%%%%%%%%%%%%
\newcommand{\jEigvecKS}[1]{\ensuremath{{\mathbf e}^{(#1)}}}   % jacobiam eigenvector, redefined here to avoid conflict with chaosbook notation. Used in ksStSp chapter.


%%%%%%%%%%%%% Operators %%%%%%%%%%%%%%%%%%%%%%%
\newcommand{\PperpOp}{\mathbf{P}^{\perp}}
\newcommand{\Pperp}{P^{\perp}}

%%%%%%%%%%%%%% Penalizing loops %%%%%%%%%%%%%%%%%

\newcommand\fp[2]{{\frac{\partial #1}{\partial #2}}}
\newcommand\fder[2]{{\frac{d #1}{d #2}}}
\newcommand\fsd[2]{{d #1/d #2}}
\newcommand\fsp[2]{{\partial #1/\partial #2}}
\newcommand\fps[3]{{\frac{\partial^2 #1}{\partial #2 \partial #3}}}
\newcommand\fsps[3]{{\partial^2 #1/\partial #2 \partial #3}}

\newcommand\Js{\mathbf{\tilde{J}}}
\newcommand\JL{\mathbf{\tilde{J}}_L}
\newcommand\Jp{\mathbf{J}_p}

\newcommand\dtds{\frac{v.\tilde{v}}{v^2}}
\newcommand\hdtds{\frac{u.\tilde{u}}{u^2}}


%%%%%%%%%%%%%%%%%%%%%%%%%%%%%%%%%%%%%%%%%%%%%%%%%%%%
