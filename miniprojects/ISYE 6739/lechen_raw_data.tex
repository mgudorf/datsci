The COVID-19 Open Research dataset~\cite{covid19weather} is used in the study of COVID-19's spread. This dataset consists of there parts:
\begin{inparaenum}[a)]
    \item COVID-19 confirmed cases and fatalities group by regions (states), 
    \item geography data (e.g., latitude and longitude), and
    \item 9 categories of weather data imported from NOAA GSOD dataset~\cite{weatherdata}.
\end{inparaenum}
It records the confirmed cases from \date{2020-01-22} through \date{2020-04-11}. To reason about the spread of COVID-19 in China, we analyzed the average growth rate and daily growth rate in each region, via a number of statistical methods. The average growth rate and series of daily growth rates are calculated based on the number of confirmed cases. The normality of daily growth rates is tested by D'Agostino's K-squared test~\cite{normaltest}. For each region's average growth rate, two analysis was conducted:
\begin{inparaenum}[a)]
    \item using covariance to analyze the relationship between the growth rate and distance to Hubei (the outbreak province of COVID-19), and 
    \item using ANOVA to compare the growth rates in distinct environments.
\end{inparaenum}
Through these analysis, we try to figure out potential factors that have a positive/negative effect on the spread of COVID-19.
