\subsection{Mike Badescu(3) Wednesday, September 18th 2019}

Writing version two and later; work on version two

Don't need to reinvent the wheel and can use.

James Hamilton time series. Forecasting package in R package.
(guy from australia?) Look over this book. Uses practical
things on how to do things in real life. Hamilton's book
covers many theory.

How should I balance my studies; application vs. theory?
Best thing is to be strict with yourself; losing something
by not studying but there.

Tip from Mike regarding discipline. Describe
what I'm doing rather; bring the document to the
state of code.

Writing down new ideas satisfies the brain.
Learning is going down the rabbit holes.


Data science C.V. work on projects that require
data wrangling.

Many ways of doing stochastic calculus and applying
models to data; do this is if you want.
Work in the light. Choose data and then choose question.

Much easier to collect similar data sets then collect data
in a new way. How would you implement this in practice
to reproduce a study. Just work with the data.

Should work on a project; build based on previous experiences.
Build based on other projects. Do not worry about picking
the wrong data set; just work.

This is my data sets
this is what i predict
this is how what i am going to measure
need to join table
methods.

Have to-do list instead of refactoring. Don't create
environments; have to explain the setup so don't make
it too complicated. environments are more important when
developing things and employ it in production; want it
to be as close as possible to deployment environment.

If you deploy code in a docker container run as root.
Create a virtual environment in docker as well; do not
rely on the distribution.

Focus on jupyter and getting things up and running;
save docker to later. Take things from project and
produce a ``data product''. Pick the low hanging fruit.

Very quick way to deal with code style; have any color
as long as its black. Black is a formatter for python code.
